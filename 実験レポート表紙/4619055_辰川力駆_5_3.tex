\documentclass[12pt]{jarticle}
\usepackage{TUSIReport}
\usepackage{otf}
\usepackage[dvipdfmx]{graphicx}
\usepackage[dvipdfmx]{color}
\usepackage{amsmath}
\usepackage{amssymb}
\usepackage{color}
\usepackage{hhline}
\usepackage{fancybox,ascmac}
\usepackage{multirow}
\usepackage{url}
\usepackage{bm}
\usepackage{listings,jlisting}

\begin{document}
%%%%%%%%%%%%%%%%%%%%%%%%%%%%%%%%%%%%%%%%%%%%%%%%%%%%%%%%%%%%%%
% 表紙を出力する場合は,\提出者と\共同実験者をいれる
% \提出者{科目名}{課題名}{提出年}{提出月}{提出日}{学籍番号}{氏名}
% \共同実験者{一人目}{二人目}{..}{..}{..}{..}{..}{八人目}
%%%%%%%%%%%%%%%%%%%%%%%%%%%%%%%%%%%%%%%%%%%%%%%%%%%%%%%%%%%%%%
\提出者{情報工学実験2}{実験テーマ5 教育システム設計}{2020}{11}{16}{4619055}{辰川力駆}
\共同実験者{書かないと!}{}{}{}{}{}{}{}

%%%%%%%%%%%%%%%%%%%%%%%%%%%%%%%%%%%%%%%%%%%%%%%%%%%%%%%%%%%%%%
% 表紙を出力しない場合は,以下の「\表紙出力」をコメントアウトする
%%%%%%%%%%%%%%%%%%%%%%%%%%%%%%%%%%%%%%%%%%%%%%%%%%%%%%%%%%%%%%
\表紙出力

%%%%%%%%%%%%%%%%%%%%%%%%%%%%%%%%%%%%%%%%%%%%%%%%%%%%%%%%%%%%%%
% 以下はレポート本体である.別途 TeXファイルを作成し \input 使っても良い
%%%%%%%%%%%%%%%%%%%%%%%%%%%%%%%%%%%%%%%%%%%%%%%%%%%%%%%%%%%%%%

\section{要旨}


\section{目的}
統計モデルを用いた分析は、例えば商品の推薦や迷惑メールの削除機能など、
身近な機能を支える基本的な技術となっている。
本実験では、このような統計モデルを用いた分析に欠かせない、
パラメタの推定の方法について、
基本的な技術を習得することを目的とする。

\section{理論}
\subsection{項目反応理論:複数問の反応からの能力値推定}
複数の問題に対する正誤反応を得た場合の能力値の推定について考える。
今、
ある問題系列について正誤反応${\bf X}_j=(x_{1,j},x_{2,j},...,x_{n,j})$が
与えられたとする。
この時の能力値$\theta$の推定値を考える際には以下のような数式を考えれば良い。
\begin{eqnarray}
    \hat{\theta}&=& \mathop{\rm arg~max}\limits_{\theta} P(\theta \mid {\bf X}_j)\\
    &=& \mathop{\rm arg~max}\limits_{\theta} P(\theta)P({\bf X}_j \mid \theta)
\end{eqnarray}
ここで$P({\bf X}_j \mid \theta)$はある能力値$\theta$の受験者が、
それぞれの項目に${\bf X}_j)$のように反応する確率である。
そのため、それぞれの項目への正誤が$\theta$のみに影響され定まるとすれば、
$P({\bf X}_j \mid \theta)$は以下のようにかける。
\begin{eqnarray}
    P({\bf X}_j \mid \theta)&=&P(x_{1,j} \mid \theta)\times P(x_{2,j} \mid \theta) \times \cdots \times P(x_{n,j} \mid \theta)\\
    &=& \prod_{i=1}^n P(x_{i.j}\mid \theta)
\end{eqnarray}
つまり、複数のサイコロを同時に投げたときと同様に同時確率と見なすことができる。
それぞれの項目への能力値$\theta$だけを媒介に独立に正誤反応していると考える。
これを局所独立過程という。
すなわち、それぞれの項目が他の項目の小rたえやヒントになっていない状況である。

また、$P(x_{i,j}\mid \theta)$は正答の場合と
誤答の場合両方を以下のように表すことができる。
\begin{eqnarray}
    P(x_{i,j}\mid \theta)=P_i(\theta)^{x_{i,j}}\times (1-P_i(\theta))^{1-x_{i,j}}
\end{eqnarray}
従って考えるべき式は以下のようになる。
\begin{eqnarray}
    \hat{\theta}&=& \mathop{\rm arg~max}\limits_{\theta} P(\theta \mid {\bf X}_j)\\
    &=& \mathop{\rm arg~max}\limits_{\theta} P(\theta)\prod_{i=1}^n P(x_{i,j} \mid \theta)\\
    &=& \mathop{\rm arg~max}\limits_{\theta} P(\theta)\prod_{i=1}^n (P_i(\theta)^{x_{i,j}}(1-P_i(\theta))^{1-x_{i,j}})
\end{eqnarray}
ただし、これを計算機により計算機により計算することは、
値域がをとる関数を複数回かけることになり、
計算誤差が生じやすい。
そのため、実装上は、このような関数の対数関数を考える。
${\rm log}$は単調増加関数であり、
ある関数$f(x)=x$が$x=x_{\rm max}$で最大を取るとき、
${\rm log}(f(x))$も$x=x_{\rm max}$で最大を取る。
そのため、以下が成り立つ。
\begin{eqnarray}
    \hat{\theta}&=& \mathop{\rm arg~max}\limits_{\theta} P(\theta \mid {\bf X}_j)\\
    &=& \mathop{\rm arg~max}\limits_{\theta} \ln{(P(\theta \mid {\bf X}_j))}
\end{eqnarray}
そのため,実装上は以下を計算すれば良い.
\begin{eqnarray}
    \hat{\theta}=\mathop{\rm arg~max}\limits_{\theta} \Bigl\{\ln{(P(\theta))+\sum_{i=1}^n \Bigl(x_{i,j}\ln(P_i(\theta))+(1-x_{i,j})\ln(1-P_i(\theta))\Bigr)}\Bigr\}
\end{eqnarray}
これを対数尤度関数と呼ぶ.

\clearpage
\section{課題}
\subsection{課題2-1}
\begin{shadebox}
    課題1-3で解いた項目について、項目反応関数の概形を描く。
\end{shadebox}


\begin{table}[htb]
    \begin{center}
        \caption{解いた問題の特性パラメタ}
        \begin{tabular}{|c|r|r|}
            \hline
            問題 & $aパラメタ$ & $bパラメタ$ \\
            \hline
            3    & 2.87168     & 0.69892     \\
            9    & 0.42082     & 0.22627     \\
            21   & 1.03497     & 0.31148     \\
            24   & 0.71798     & 1.22817     \\
            30   & 1.20718     & 0.65633     \\
            35   & 0.72641     & 0.14052     \\
            40   & 0.31796     & 2.23511     \\
            51   & 0.51611     & 0.94029     \\
            \hline
        \end{tabular}
    \end{center}
\end{table}

\clearpage
\subsection{課題2-2}
\begin{shadebox}
    課題2-1で描いたグラフを参考に、それらの項目がどのような項目であったのか考察する。
\end{shadebox}

図1のグラフの中央付近の増加量について、Item3が一番正答率の変化量が大きい。
これは、$a$パラメタが他と比べて大きいからである。
つまり、Item3などの$a$パラメタが大きい問題は、
ある一定の能力値を超えるとほぼ100%の正答率となるが、
逆にその能力値を越えていないとほぼ0%の正答率となることが分かる。

$b$パラメタについては、
Item40を見ると分かるように正答率が高くなるまでに必要な能力値が高いことがわかる。
つまり、$b$パラメタが低い順に正答率が50%になるための能力値が低い。

\subsection{課題2-3}
\begin{shadebox}
    $\theta_j=b_i$の場合、正答確率は$50%$であることを証明する。
\end{shadebox}

式(1)に$\theta_j=b_i$を代入して、
\begin{eqnarray*}
    P(\theta_j=b_i)&=&\frac{1}{1+\exp\{-Da_i(\theta_j -b_i)\}}\\
    &=&\frac{1}{1+\exp\{-Da_i(b_i -b_i)\}}\\
    &=&\frac{1}{1+\exp(0)}\\
    &=&\frac{1}{2}\\
    &=&0.5
\end{eqnarray*}
よって正答確率は$50%$となる。

\subsection{課題2c-1}
\begin{shadebox}
    $\theta_j=b_i$での項目反応関数の傾きは$a_i$に比例することを証明する。
\end{shadebox}

項目反応関数の傾きを知るために、式(1)を微分すると、
\begin{eqnarray*}
    \frac{d}{d \theta_j}P(\theta_j)&=& \frac{Da_i\exp\{-Da_i(\theta_j -b_i)\}}{[1+\exp\{-Da_i(\theta_j -b_i)\}]^2}
\end{eqnarray*}
この式に$\theta_j=b_i$を代入して、
\begin{eqnarray*}
    \frac{d}{d \theta_j}P(\theta_j=b_i)&=& \frac{Da_i\exp(0)}{\{1+\exp(0)\}^2}\\
    &=&\frac{1}{4}Da_i
\end{eqnarray*}
$D$は固定しているので、$a_i$に比例している。

\clearpage
\subsection{課題2–4}
\begin{shadebox}
    課題1-3で解いた項目から1題選びその正誤から描かれる事後分布のグラフを示す。
\end{shadebox}

課題1-3で解いた項目から、Item21を選んだ。
Item21の正答確率と誤答確率は図2のようになった。
もちろんだが、$正答確率+誤答確率=1$なので$確率=0.5$の直線で線対称なグラフになっている。

また、正答確率事後分布と誤答確率事後分布は図3のようになった。
正規分布は上に凸であるが、それをかけ合わせているので、上に凸なグラフとなっている。



\subsection{課題2–5}
\begin{shadebox}
    課題2-4で描いた事後分布から能力値を推定する。
    またその時の標準誤差を求める。
\end{shadebox}

まず、能力値を推定する。
Item21を正解したので、式(4)より尤度の最大値を考えて
\begin{eqnarray}
    \hat{\theta}&=&0.6 \nonumber
\end{eqnarray}
である。

次に、標準誤差を求める。
式(10)より、フィッシャー情報量は
\begin{eqnarray}
    I_i(\theta)&=&D^2a_i^2 P_i(\theta)(1-P_i(\theta)) \nonumber\\
    &=&1.7^2 \times 1.035^2\times 0.624\times (1-0.624) \nonumber\\
    &\thickapprox& 0.726 \nonumber
\end{eqnarray}
であるから、式(11)より標準誤差は
\begin{eqnarray}
    {\rm se}(\hat{\theta})&=&I_i(\hat{\theta})^{-\frac{1}{2}} \nonumber\\
    &=&\frac{1}{\sqrt{I_i(\hat{\theta})}}\nonumber\\
    &\thickapprox&1.174\nonumber
\end{eqnarray}
である。

\clearpage
\subsection{課題2–6}
\begin{shadebox}
    $\theta$の刻み幅を0.1以外に2パターンを自ら適当に変更した場合の
    能力値と標準誤差の結果を記述し、結果について考察する。
\end{shadebox}

$\theta$の刻み幅を0.05と0.8にした場合を考えた。
図4と図5は、$\theta$の刻み幅を0.05にした場合で、
図6と図7は、$\theta$の刻み幅を0.8にした場合である。


\subsubsection*{$\theta$の刻み幅0.05}
能力値を推定する。
Item21を正解したので、式(4)より尤度の最大値を考えて
\begin{eqnarray}
    \hat{\theta}&=&0.65 \nonumber
\end{eqnarray}
である。

次に、標準誤差を求める。
式(10)より、フィッシャー情報量は
\begin{eqnarray}
    I_i(\theta)&=&D^2a_i^2 P_i(\theta)(1-P_i(\theta)) \nonumber\\
    &=&1.7^2 \times 1.035^2\times 0.645\times (1-0.645) \nonumber\\
    &\thickapprox& 0.709 \nonumber
\end{eqnarray}
であるから、式(11)より標準誤差は
\begin{eqnarray}
    {\rm se}(\hat{\theta})&=&I_i(\hat{\theta})^{-\frac{1}{2}} \nonumber\\
    &=&\frac{1}{\sqrt{I_i(\hat{\theta})}}\nonumber\\
    &\thickapprox&1.188\nonumber
\end{eqnarray}
である。

\clearpage
\subsubsection*{$\theta$の刻み幅0.8}
能力値を推定する。
Item21を正解したので、式(4)より尤度の最大値を考えて
\begin{eqnarray}
    \hat{\theta}&=&0.8 \nonumber
\end{eqnarray}
である。

次に、標準誤差を求める。
式(10)より、フィッシャー情報量は
\begin{eqnarray}
    I_i(\theta)&=&D^2a_i^2 P_i(\theta)(1-P_i(\theta)) \nonumber\\
    &=&1.7^2 \times 1.035^2\times 0.703\times (1-0.703) \nonumber\\
    &\thickapprox& 0.647 \nonumber
\end{eqnarray}
であるから、式(11)より標準誤差は
\begin{eqnarray}
    {\rm se}(\hat{\theta})&=&I_i(\hat{\theta})^{-\frac{1}{2}} \nonumber\\
    &=&\frac{1}{\sqrt{I_i(\hat{\theta})}}\nonumber\\
    &\thickapprox&1.243\nonumber
\end{eqnarray}
である。

$\theta$の刻み幅0.05および$\theta$の刻み幅0.8の
能力値と標準誤差は課題2-5で求めた値と違っていた。
これは、能力値を求める際に、使用できる$\theta$が違うからである。

$\theta$の刻み幅を小さくすれば、さらに使える能力値が多くなるので、
正確に推定するならば刻み幅を小さくすればよいと考える。

\clearpage
\subsection{課題2–7}
\begin{shadebox}
    能力値の事前分布が平均-0.3、分散1の正規分布に従うと仮定された場合、
    自分の能力値と情報量について計算し、結果について考察する。
\end{shadebox}
能力値の事前分布が平均-0.3、分散1の正規分布に従うと仮定された場合、
正答確率と誤答確率のグラフは変わらないので図2と同じになる。
事前分布が変わると影響されるのは正答確率事後分布と誤答確率事後分布である。
そのグラフを図8に表した。

能力値を推定する。
Item21を正解したので、式(4)より尤度の最大値を考えて
\begin{eqnarray}
    \hat{\theta}&=&0.5 \nonumber
\end{eqnarray}
である。

次に、標準誤差を求める。
式(10)より、フィッシャー情報量は
\begin{eqnarray}
    I_i(\theta)&=&D^2a_i^2 P_i(\theta)(1-P_i(\theta)) \nonumber\\
    &=&1.7^2 \times 1.035^2\times 0.582\times (1-0.582) \nonumber\\
    &\thickapprox& 0.753 \nonumber
\end{eqnarray}
であるから、式(11)より標準誤差は
\begin{eqnarray}
    {\rm se}(\hat{\theta})&=&I_i(\hat{\theta})^{-\frac{1}{2}} \nonumber\\
    &=&\frac{1}{\sqrt{I_i(\hat{\theta})}}\nonumber\\
    &\thickapprox&1.152\nonumber
\end{eqnarray}
である。

図3と図8のグラフを比べると分かるが、
全体的に少しだけ左の値が大きくなっている。
今回は誤答確率事後分布が-0.3付近で山なりになっているので、
その部分が大きくなっている。
それらによって、能力値やフィッシャー情報量、
標準誤差が変わっている。


\section{まとめ}
項目反応理論を用いた場合の受験者の能力値の推定について実験を行った。
また、課題を通して刻み幅を変更した場合の能力値の推定や、
標準正規分布ではなく違った正規分布に従う場合の推定も行った。
これらを通して項目反応理論を理解することができた。


\clearpage
% 付録
\appendix
%%%%%%%%%%%%%%%%%%%%%%%%%%%%%%%%%%%%%%%%%%%%%%%%%%%%%%%%%%%%%%
\end{document}