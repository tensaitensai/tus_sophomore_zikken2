\documentclass[12pt]{jarticle}
\usepackage{TUSIReport}
\usepackage{otf}
\usepackage[dvipdfmx]{graphicx}
\usepackage[dvipdfmx]{color}
\usepackage{amsmath}
\usepackage{amssymb}
\usepackage{color}
\usepackage{hhline}
\usepackage{fancybox,ascmac}
\usepackage{multirow}
\usepackage{url}
\usepackage{bm}
\usepackage{listings,jlisting}

\begin{document}
%%%%%%%%%%%%%%%%%%%%%%%%%%%%%%%%%%%%%%%%%%%%%%%%%%%%%%%%%%%%%%
% 表紙を出力する場合は,\提出者と\共同実験者をいれる
% \提出者{科目名}{課題名}{提出年}{提出月}{提出日}{学籍番号}{氏名}
% \共同実験者{一人目}{二人目}{..}{..}{..}{..}{..}{八人目}
%%%%%%%%%%%%%%%%%%%%%%%%%%%%%%%%%%%%%%%%%%%%%%%%%%%%%%%%%%%%%%
\提出者{情報工学実験2}{実験テーマ5 教育システム設計}{2020}{11}{9}{4619055}{辰川力駆}
\共同実験者{}{}{}{}{}{}{}{}

%%%%%%%%%%%%%%%%%%%%%%%%%%%%%%%%%%%%%%%%%%%%%%%%%%%%%%%%%%%%%%
% 表紙を出力しない場合は,以下の「\表紙出力」をコメントアウトする
%%%%%%%%%%%%%%%%%%%%%%%%%%%%%%%%%%%%%%%%%%%%%%%%%%%%%%%%%%%%%%
\表紙出力

%%%%%%%%%%%%%%%%%%%%%%%%%%%%%%%%%%%%%%%%%%%%%%%%%%%%%%%%%%%%%%
% 以下はレポート本体である.別途 TeXファイルを作成し \input 使っても良い
%%%%%%%%%%%%%%%%%%%%%%%%%%%%%%%%%%%%%%%%%%%%%%%%%%%%%%%%%%%%%%

\section{要旨}
\section{目的}

\section{理論}
\subsection{項目反応理論基礎}
項目反応理論では、
ある受験者$j$がある項目$i$に正答する$x_{i,j}=1$確率を以下の式でモデル化する。
\begin{equation}
    P(x_{i,j}=1|\theta_j,a_i,b_i)=P_i(x_{i,j}=1|\theta_j)=P_i(\theta_j)=\frac{1}{1+\exp\{-Da_i(\theta_j -b_i)\}}
\end{equation}
ここで$\theta_j$は受験者の能力値パラメタ、
$a_i$は項目の識別力パラメタ、
$b_i$は項目の困難度パラメタである。
また、$D=1.7$としたとき、このモデルは以下の特徴がある。
\begin{enumerate}
    \item $\theta_j=b_i$の場合、正答確率は50$\%$となる。
    \item $\theta_j=b_i$での正答率の傾きは$a_i$に比例する。
    \item $a_i=1,b_i=0$のとき、この関数は累積標準正規分布の良い近似となる。
\end{enumerate}

$a$は識別力パラメタと呼ばれる。
$a$が0に近い項目は、
能力値によらず一定の正答率となるような、
能力に関係ない項目となる。
$a$が$0$に近い値の時は、能力値に関わらず正答率が高い傾向がある。

$b$は困難度パラメタと呼ばれる。
項目反応関数の正答確率が$50%$となる点の$\theta$と同じ値となる。
$b$が大きな項目では正答に必要な能力値が大きくなる。
$b$が大きいほど、小さい能力値での正答率が低い傾向がある。

\subsection{最尤推定を用いた受験者能力の推定}
項目反応理論を用いた受験者の能力値の推定も最尤推定と同様に計算する。
例えば、今ある問題に正答した$(x_{i,j}=1)$とする。
この時の受験者の能力値は以下の数式を考えれば良い。
\begin{equation}
    \hat{\theta}=\mathop{\rm arg~max}\limits_{\theta}P(\theta|x_{i,j}=1)
\end{equation}
ただし、
$P(\theta_j|x_{i,j}=1)$はIRTのモデルに従って直接は与えられていないため、
ベイズの定理より、$P_i(x_{i,j}=1|\theta_j)$を用いて考える。
\begin{eqnarray}
    P(\theta_j|x_{i,j}=1)=\frac{P(\theta_j)}{P(x_{i,j}=1)}P_i(x_{i,j}=1|\theta_j)
\end{eqnarray}
ここで、
$P(x_{i,j}=1)$はその問題に正答できる確率であるが、
変数$\theta_j$と独立な変数であるため、
$\mathop{\rm arg~max}\limits_{\theta}$を考える上で定数とみなすことができる。
従って、以下の式を考えても結果は変わらない。
\begin{eqnarray}
    \hat{\theta}&=&\mathop{\rm arg~max}\limits_{\theta}P(\theta|x_{i,j}=1) \\
    &=&\mathop{\rm arg~max}\limits_{\theta}P(\theta)P_i(x_{i,j}=1|\theta_j)
\end{eqnarray}
$P(\theta)$は能力値が一般にどのような分布をしているかを表すので、
標準正規分布していると仮定できる。
\begin{eqnarray}
    P(\theta)=\frac{1}{\sqrt{2\pi}}\exp \left(-\frac{\theta^2}{2}\right)
\end{eqnarray}
また、
誤答した場合も同様に以下のように考えることができる。
\begin{eqnarray}
    \hat{\theta} &=& \mathop{\rm arg~max}\limits_{\theta}P(\theta|x_{i,j}=0)\\
    &=& \mathop{\rm arg~max}\limits_{\theta}P(\theta)P_i(x_{i,j}=0|\theta_j)\\
    &=& \mathop{\rm arg~max}\limits_{\theta}P(\theta)(1- P_i(x_{i,j}=1|\theta_j))
\end{eqnarray}

このような関数を考えることで、
ある問題に正答、
あるいは誤答した場合の受験者の能力値を推定することが可能である。
また、$P(\theta)$は問題に正答したという事実を受け取る前の確率であり、
事前確率と呼ばれることがある。
加えて、$P(\theta|x_{i,j})$は問題に正答あるいは
誤答したという事実を受け取ったあとの確率なので、
事後確率と呼ばれることがある。

また、項目反応理論では、
項目情報量関数と呼ばれる指標が非常に重要となる。
項目情報量関数とは、
項目反応関数について項目への反応から能力値を推定する際のフィッシャー情報量を
算出したものであり、2パラメタロジスティックモデルでは以下のような式となる。
\begin{eqnarray}
    I_i(\theta)=D^2a_i^2 P_i(\theta)(1-P_i(\theta))
\end{eqnarray}
フィッシャー情報量は一般に$\hat{\theta}$の標準誤差${\rm se}(\hat{\theta})$
と以下の関係を持つ
\begin{eqnarray}
    {\rm se}(\hat{\theta})=I_i(\hat{\theta})^{-\frac{1}{2}}=\frac{1}{\sqrt{I_i(\hat{\theta})}}
\end{eqnarray}

そのため、この項目情報量を用いることで、
推定された$\hat{\theta}$に対してどの程度標準誤差があるかを見積もることができる。

\section{課題}
\section{まとめ}

% 参考文献
\begin{thebibliography}{99}
    \label{sannkoubunnkenn_chapter}
\end{thebibliography}

\clearpage
% 付録
\appendix
\section{付録}
%%%%%%%%%%%%%%%%%%%%%%%%%%%%%%%%%%%%%%%%%%%%%%%%%%%%%%%%%%%%%%
\end{document}