\documentclass[12pt]{jarticle}
\usepackage{TUSIReport}
\usepackage{otf}
\usepackage{ascmac}
\usepackage{listings,jlisting}
\usepackage{url}
\usepackage[dvipdfmx]{graphicx}
\usepackage{here}
\usepackage{subfigure}
\usepackage{amssymb}
\usepackage{multirow}
\usepackage{longtable}
\begin{document}
%%%%%%%%%%%%%%%%%%%%%%%%%%%%%%%%%%%%%%%%%%%%%%%%%%%%%%%%%%%%%%
% 表紙を出力する場合は,\提出者と\共同実験者をいれる
% \提出者{科目名}{課題名}{提出年}{提出月}{提出日}{学籍番号}{氏名}
% \共同実験者{一人目}{二人目}{..}{..}{..}{..}{..}{八人目}
%%%%%%%%%%%%%%%%%%%%%%%%%%%%%%%%%%%%%%%%%%%%%%%%%%%%%%%%%%%%%%
\提出者{情報工学実験2}{実験テーマ5 教育システム設計}{2020}{9}{21}{4619023}{加藤零}
\共同実験者{}{}{}{}{}{}{}{}

%%%%%%%%%%%%%%%%%%%%%%%%%%%%%%%%%%%%%%%%%%%%%%%%%%%%%%%%%%%%%%
% 表紙を出力しない場合は,以下の「\表紙出力」をコメントアウトする
%%%%%%%%%%%%%%%%%%%%%%%%%%%%%%%%%%%%%%%%%%%%%%%%%%%%%%%%%%%%%%
\表紙出力

%%%%%%%%%%%%%%%%%%%%%%%%%%%%%%%%%%%%%%%%%%%%%%%%%%%%%%%%%%%%%%
% 以下はレポート本体である.別途 TeXファイルを作成し \input 使っても良い
%%%%%%%%%%%%%%%%%%%%%%%%%%%%%%%%%%%%%%%%%%%%%%%%%%%%%%%%%%%%%%

\section{要旨}
\section{目的}

\section{理論}
\section{課題}
\subsection{課題1-1}


\subsection{課題1-2}
\begin{shadebox}
    \quad この大学にある経営学部学生がいた場合,その性別を推定せよ.
\end{shadebox}
\vspace{\baselineskip}
本質的には課題1-1と同一なので同様の計算式により解を求める.
式5を応用することで以下の式を導くことができる.
\begin{eqnarray}
    (\hat{性別})&=& \mathop{\rm arg~max}\limits_{性別} P(性別 \mid 経営学部生) \nonumber\\
    &=&\mathop{\rm arg~max}\limits_{性別}\frac{P(性別)}{P(経営学部生)}P(経営学部生 \mid 性別)
\end{eqnarray}
また,$P(経営学部生)$は定数として扱うことができるため,式6を応用してこの式は実質的に$P(性別)P(経営学部生 \mid 性別)$として扱うことができ,これを尤度とする.
結果は以下のようになる.
\begin{table}[h]
    \centering
    \caption{性別に関する確率と尤度}
    \begin{tabular}{|c|r|r|} \hline
        性別 & \multicolumn{1}{|c|}{確率} & \multicolumn{1}{|c|}{尤度}       \\
             & $P(性別 \mid 経営学部生)$  & $P(性別)P(経営学部生 \mid 性別)$ \\
        \hline\hline
        男性 & 0.683                      & 0.057                            \\ \hline
        女性 & 0.317                      & 0.027                            \\ \hline
    \end{tabular}
\end{table}

課題1-1において尤度の妥当性は示されているため,表より男性である可能性が高いと言える.

\subsection{課題1-3}
\begin{shadebox}
    \quad 配布された問題を4題以上ランダムに選び解答せよ.
\end{shadebox}
\vspace{\baselineskip}
解答結果は次のようになった.
\clearpage
\begin{table}[h]
    \centering
    \caption{各学部に関する確率と尤度}
    \begin{tabular}{|c|r|r|r|} \hline
        問題 & \multicolumn{1}{|c|}{解答} & \multicolumn{1}{|c|}{正誤} & \multicolumn{1}{|c|}{正答} \\
        \hline\hline
        1    & 2                          & ○                          &                            \\ \hline
        4    & 1                          & ○                          &                            \\ \hline
        7    & 5                          & ○                          &                            \\ \hline
        10   & 1                          & ○                          &                            \\ \hline
        13   & 4                          & ○                          &                            \\ \hline
        16   & 5                          & ○                          &                            \\ \hline
        19   & 5                          & ○                          &                            \\ \hline
        22   & 2                          & ○                          &                            \\ \hline
        25   & 2                          & ○                          &                            \\ \hline
        28   & 2                          & ○                          &                            \\ \hline
    \end{tabular}
\end{table}
どの問題も比較的簡単であったため全問正解となってしまった.時間制限があれば結果は変わったと思われる.
\subsubsection{問題1}
問題文を$3,2,1$順に解釈していくと$お酒が好きな人 \Rightarrow 毎日歯をみがく
    \Rightarrow 風呂が好きである \Rightarrow きれい好きである$が成立するため2が正答となる.
\subsubsection{問題4}
問題文の1つ目と3つ目の文章より$Dである \Rightarrow Bである \Rightarrow Aである$が成立する.
この図式は実質的に$Dである \Rightarrow Aである$と等価であり,対偶を取ると$Aでない \Rightarrow Bでない$となるため1が正答である.
\subsubsection{問題7}
4つ目の文章に着目すると,「物理が好きな者の中には芸術が好きな者がいる.」とあるが他に芸術に関して言及している文章は存在しない.
ゆえに芸術が好きかどうかは他の科目に依存しないことがわかる.
また,2つ目の文章において「社会が好きな者は国語が好きである.」とあるため,社会と国語を同時に好むものがいることがわかる.
\subsubsection{問題10}
AとCが前半に真実を述べており後半に嘘を述べているとし,BとDが前半に嘘を述べており後半に真実を述べているとするとAとCの発言よりそれぞれ音楽家と画家であるということが確定する.
Dの発言またはAの発言によりDは作家でないことが確定するため,消去法によりDは詩人となる.
\subsubsection{問題13}
それぞれの発言は誰がうそつきかどうかの二元論なので,うそつきを仮定して発言を否定すれば真の発言を取得できる.
このときBとCをうそつきだと仮定すると,AとDの発言に矛盾はないことがわかる.また,BとCの発言を否定することで「Cはうそつきである」という発言と「Dはうそつきでない」という発言を得ることができ,
これらの発言は題に矛盾しない.故に4が正答となる.
\subsubsection{問題16}
これらの全員が真実を述べていると仮定すると矛盾が生じるが,そのときにBの発言を否定すると矛盾が解消される.故に5が正答となる.
\subsubsection{問題19}
「不可だった」という発言に着目する.このとき,Dの発言を真だとするとBが不可でAは可でないことがわかる.
このときAを不可だと仮定するとCは不可になり、消去法でEは真の発言を述べていることがわかる.
以上の発言は矛盾しなため,Eが可であることがわかる.故に5が正答となる.
\subsubsection{問題22}
問題文のアルファベットを4文字前にするとRYOUSEIRUIHADOREKAになるため両生類を答えれば良い.
故に2が正答となる.
\subsubsection{問題25}
問題文によるとさいたまは「75946535」でとつとりは「61636114」はであるため,これらの数字列を2桁づつ区切って五十音表に当てはめれば良い.
すると,問題文は水曜日の前となるため正答は2となる.
\subsubsection{問題28}
問題文によると夕焼けは「8C1C8E2B」で赤とんぼは「1E2E4A11E6AA」はである.
数字は五十音表における列数を右から示しており,アルファベットは行数を下から示している.
濁点はアルファベットを二回繰り返すことで表現している.
すると,問題文は水曜日の前となるため正答は2となる.
\subsection{課題1-4}
\begin{shadebox}
    \quad
    課題1-3で解いた問題群を,1問あたり(100/問題数)点の100点満点のテストとし,
    自分の偏差値と順位を求めよ.(平均点は60,標準偏差15,1000人中の順位で,1000人の得点分布は正規分布
    に従っているとする.また,計算過程も記述すること)
\end{shadebox}
\vspace{\baselineskip}
ある受験者の偏差値$S$は平均点$u_x$,受験者の得点$x$,標準偏差$\sigma_x$を用いると以下のように表せる.
\begin{equation}
    S = \frac{10(x-u_x)}{\sigma_x}+50
\end{equation}
題の条件と課題1-3の結果$x=100$を代入すると$S=76.6666666\cdots\simeq76.7$となる.
題よりこの得点は正規分布に従っているため,標準化して標準正規分布表を扱って順位を算出しても良かったが,今回はExcelのNORM.DIST関数を用いた.1-NORM.DIST(100,60,15,TRUE)*1000とすると$3.83038057\cdots$となり3位以上に位置することがわかる.

\subsection{課題1-5}
\begin{shadebox}
    \quad
    世の中にいる「あすか」という名前の人の男女比が2:8だとする.この大学に「あすか」という人が1名所属していることがわかっているとき,「あすか」さんの学部と性別を上位3つまで推定せよ.
\end{shadebox}
\vspace{\baselineskip}
まず,題の前提を用いることで「あすか」という名前の人の性別が男性及び女性になる確率は以下のようになる
\begin{eqnarray}
    P(男性\mid あすか) &=& 0.2\\
    P(女性\midあすか) &=& 0.8
\end{eqnarray}
また,男子学生がそれぞれの学部に所属している確率に及びある女子学生がそれぞれの学部に所属している確率に式10,11をかけたものが今回用いる尤度であり以下になる.
\begin{eqnarray*}
    (\hat{学部})&=& \mathop{\rm arg~max}\limits_{学部} P(学部 \mid 男子学生)P(男性\midあすか) \\
    (\hat{学部})&=& \mathop{\rm arg~max}\limits_{学部} P(学部 \mid 女子学生)P(男性\midあすか) \\
\end{eqnarray*}
つまり,以下の式となる.
\begin{eqnarray*}
    \mathop{\rm arg~max}\limits_{学部}P(学部 \mid 性別)P(性別\midあすか)\\
\end{eqnarray*}
結果は以下のようになる.
\begin{table}[h]
    \centering
    \caption{各学部に関する確率と尤度}
    \begin{tabular}{|c|r|r|r|r|} \hline
        学部         & \multicolumn{2}{|c|}{学生数} & \multicolumn{1}{|c|}{男性} & \multicolumn{1}{|c|}{女性}         \\ \cline{2-3}
                     & 男子学生数                   & 女子学生数                 &                            &       \\ \hline \hline
        理学部第一部 & 2230                         & 611                        & 0.035                      & 0.129 \\ \hline
        理学部第二部 & 1271                         & 390                        & 0.020                      & 0.082 \\ \hline
        薬学部       & 492                          & 547                        & 0.008                      & 0.115 \\ \hline
        工学部       & 1771                         & 417                        & 0.028                      & 0.088 \\ \hline
        工学部第二部 & 701                          & 148                        & 0.011                      & 0.031 \\ \hline
        理工学部     & 4289                         & 870                        & 0.067                      & 0.183 \\ \hline
        基礎工学部   & 1028                         & 372                        & 0.016                      & 0.078 \\ \hline
        経営学部     & 950                          & 441                        & 0.015                      & 0.092 \\ \hline
        学部合計     & 12732                        & 3796                       & 0.200                      & 0.800 \\ \hline
    \end{tabular}
\end{table}
以上より,「あすか」という名前の人の学部・性別上位3位は
\begin{enumerate}
    \item 理工学部・女子学生
    \item 理学部第一部・女子学生
    \item 薬学部・女子学生
\end{enumerate}
である可能性が高い.

\section{まとめ}
今回の演習・課題においてある大学における男女別学部別の在籍者数をモデルとして扱うことで,高等学校で学習したベイズの定理を拡張し,事前分布から事後分布を推定することができることや大小関係の比較における尤度の有用性を再確認できた.

% 参考文献
\begin{thebibliography}{99}
    \label{sannkoubunnkenn_chapter}
    \bibitem{} 高校数学の美しい物語\\
    \url{https://mathtrain.jp/bayesinfer}\\
    最終閲覧日; 2020/9/26
\end{thebibliography}

%%%%%%%%%%%%%%%%%%%%%%%%%%%%%%%%%%%%%%%%%%%%%%%%%%%%%%%%%%%%%%
\end{document}