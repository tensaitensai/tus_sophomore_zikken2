\documentclass[12pt]{jarticle}
\usepackage{TUSIReport}
\usepackage{otf}
\usepackage{graphicx}
\usepackage{amsmath}
\usepackage{amssymb}
\usepackage{color}
\usepackage{hhline}
\usepackage{fancybox,ascmac}
\usepackage{multirow}
\usepackage{url}
\usepackage{listings,jlisting}
\lstdefinestyle{log}{
    frame={tblr},
    basicstyle={\footnotesize},
    tabsize={4},
}
\lstdefinestyle{lstR}{
    language={C},
    backgroundcolor={\color[gray]{.85}},
    basicstyle={\small},
    identifierstyle={\small},
    commentstyle={\small\ttfamily \color[rgb]{0,0.5,0}},
    keywordstyle={\small\bfseries \color[rgb]{1,0,0}},
    ndkeywordstyle={\small},
    stringstyle={\small\ttfamily \color[rgb]{0,0,1}},
    frame={tb},
    breaklines=true,
    columns=[l]{fullflexible},
    numbers=left,
    xrightmargin=0zw,
    xleftmargin=3zw,
    numberstyle={\scriptsize},
    stepnumber=1,
    numbersep=1zw,
    morecomment=[l]{//}
}
\begin{document}
%%%%%%%%%%%%%%%%%%%%%%%%%%%%%%%%%%%%%%%%%%%%%%%%%%%%%%%%%%%%%%
% 表紙を出力する場合は,\提出者と\共同実験者をいれる
% \提出者{科目名}{課題名}{提出年}{提出月}{提出日}{学籍番号}{氏名}
% \共同実験者{一人目}{二人目}{..}{..}{..}{..}{..}{八人目}
%%%%%%%%%%%%%%%%%%%%%%%%%%%%%%%%%%%%%%%%%%%%%%%%%%%%%%%%%%%%%%
\提出者{情報工学実験2}{実験テーマ4 統計的推測と単回帰分析}{2020}{9}{28}{4619055}{辰川力駆}
\共同実験者{}{}{}{}{}{}{}{}

%%%%%%%%%%%%%%%%%%%%%%%%%%%%%%%%%%%%%%%%%%%%%%%%%%%%%%%%%%%%%%
% 表紙を出力しない場合は,以下の「\表紙出力」をコメントアウトする
%%%%%%%%%%%%%%%%%%%%%%%%%%%%%%%%%%%%%%%%%%%%%%%%%%%%%%%%%%%%%%
\表紙出力

%%%%%%%%%%%%%%%%%%%%%%%%%%%%%%%%%%%%%%%%%%%%%%%%%%%%%%%%%%%%%%
% 以下はレポート本体である.別途 TeXファイルを作成し \input 使っても良い
%%%%%%%%%%%%%%%%%%%%%%%%%%%%%%%%%%%%%%%%%%%%%%%%%%%%%%%%%%%%%%

\section{はじめに}
本実験では、単回帰分析の考え方と手順を理解することを目標とする。

\section{目的}
\begin{itemize}
    \item [1.]単回帰分析の考え方と手順

          単回帰分析の目的、考え方、手順を理解する
    \item [2.]単回帰分析における行列表現

          単回帰分析における行列表現(線形回帰モデル、正規方程式、最小二乗推定量など)
          を理解する
    \item [3.]実際のデータ解析

          実際のデータに回帰分析を適用することで、解析法を実践的に利用・応用できるようにする
\end{itemize}

\section{実験方法}
\subsection{実験1 単回帰分析の考え方と手順}
6つの市町村の人口と行政職員数の仮想データを表1に示す。
また、各市町村の人口を$x_i$,職員数を$y_i(i=1,...,n(=6))$と表記する。

\begin{table}[htb]
    \begin{center}
        \caption{市町村の人口と行政職員数}
        \begin{tabular}{|c|c|c|} \hline
            市町村 & 人口$x$(千人) & 職員数$y$(人) \\ \hline
            A      & 1             & 10            \\
            B      & 2             & 20            \\
            C      & 3             & 20            \\
            D      & 3             & 40            \\
            E      & 5             & 40            \\
            F      & 1             & 5             \\ \hline
            合計   & 15            & 135           \\\hline
        \end{tabular}
    \end{center}
\end{table}

\begin{itemize}
    \item [1.]次の統計量を計算する。
          \begin{eqnarray}
              \sum_{i=1}^{n} x_i, \sum_{i=1}^{n} y_i, \sum_{i=1}^{n} {x_i}^2, \sum_{i=1}^{n} {y_i}^2, \sum_{i=1}^{n} x_iy_i \nonumber
          \end{eqnarray}
    \item [2.]次式が整理することを証明する。
          \begin{eqnarray}
              \sum_{i=1}^{n} (x_i-\bar{x})^2&=&\sum_{i=1}^{n} {x_i}^2-n\bar{x}^2 \\
              \sum_{i=1}^{n} (y_i-\bar{y})^2&=&\sum_{i=1}^{n} {y_i}^2-n\bar{y}^2 \\
              \sum_{i=1}^{n} (x_i-\bar{x})(y_i-\bar{y})&=&\sum_{i=1}^{n} {x_iy_i}-n\bar{xy}
          \end{eqnarray}
    \item [3.]人口$x$と職員数$y$の基本統計量(データ数、平均、標準偏差、最小値、最大値)を計算する。
          \begin{eqnarray}
              人口xの平均&=&\bar{x}=\frac{\sum_{i=1}^{n} x_i}{n} \\
              人口xの標準偏差&=&\sqrt{\frac{\sum_{i=1}^{n} (x_i-\bar{x})^2}{n-1}}=\sqrt{\frac{\sum_{i=1}^{n} {x_i}^2-n\bar{x}^2}{n-1}} \\
              職員数yの平均&=&\bar{y}=\frac{\sum_{i=1}^{n} y_i}{n} \\
              職員数yの標準偏差&=&\sqrt{\frac{\sum_{i=1}^{n} (y_i-\bar{y})^2}{n-1}}=\sqrt{\frac{\sum_{i=1}^{n} {y_i}^2-n\bar{y}^2}{n-1}}
          \end{eqnarray}
    \item [4.]人口$x$と職員数$y$のPearsonの積率相関係数$r$を計算する。
          \begin{eqnarray}
              r=\frac{\sum_{i=1}^{n} (x_i-\bar{x})(y_i-\bar{y})}{\sqrt{\sum_{i=1}^{n} (x_i-\bar{x})^2} \sqrt{\sum_{i=1}^{n} (y_i-\bar{y})^2}}=\frac{\sum_{i=1}^{n} {x_iy_i}-n\bar{xy}}{\sqrt{\sum_{i=1}^{n} (x_i-\bar{x})^2} \sqrt{\sum_{i=1}^{n} (y_i-\bar{y})^2}}
          \end{eqnarray}
    \item [5.]人口$x$を横軸,職員数$y$を縦軸にした散布図を作成して、両者の関係を調べる。
    \item [6.]単回帰モデル$y_i=\beta_0+\beta_1 x_i + \varepsilon_i(i=1,...,n)$をあてはめる。
          $\beta_0$と$\beta_1$の推定量を$\hat{\beta_0}$と$\hat{\beta_1}$とすると、
          目的変数(応答変数)である職員数の予測値は$\hat{y_i}=\hat{\beta_0}+\hat{\beta_1}x_i$で
          与えられる。
          次式の残差平方和$S_e$を$\hat{\beta_0}$と$\hat{\beta_1}$でそれぞれ偏微分する。
          \begin{eqnarray}
              S_e=\sum_{i=1}^{n} (y_i-\hat{y_i})^2 = \sum_{i=1}^{n} (y_i-(\hat{\beta_0}+\hat{\beta_1}x_i))^2
          \end{eqnarray}
    \item [7.]正規方程式を作成する。
    \item [8.]正規方程式を解き、$\beta_0$と$\beta_1$の最小二乗推定量を数式で表現する。
    \item [9.]最小二乗推定量$\hat{\beta_0}$,$\hat{\beta_1}$の値を求める。
    \item [10.]得られた回帰直線($\hat{y_i}=\hat{\beta_0}+\hat{\beta_1}x_i$)を
          手順5で作成した散布図に図示して、結果を考察する。
    \item [11.]データ分析[回帰分析]を用いて、これまでに得られた結果と同様の結果が得られることを確認する。
\end{itemize}

\clearpage

\subsection{実験2 単回帰分析における行列表現}
データ数を$n$とする。
目的変数ベクトル$Y$と説明変数を含む定数行列$X$を次式で定義する。
\begin{enumerate}
    \item あ
\end{enumerate}


\begin{lstlisting}
    ぱあ
\end{lstlisting}
\begin{shadebox}
    \fbox{ぱあ}
\end{shadebox}
\begin{lstlisting}[caption=read\_2\_1byte.c,label=read,style = lstR]
    #include <stdio.h>
\end{lstlisting}

\section{結果・考察・課題}

\section{まとめ}
\begin{itemize}
    \item [1.]単回帰分析の考え方と手順を学んだ
          \begin{itemize}
              \item 手計算やエクセルで分析を行った
          \end{itemize}
    \item [2.]単回帰分析における行列表現を学んだ
          \begin{itemize}
              \item 実験1の手順を行列表現した
              \item Rを使い、単回帰分析を行った
          \end{itemize}
\end{itemize}

\section{感想}


\clearpage
% 参考文献
\begin{thebibliography}{99}
    \label{sannkoubunnkenn_chapter}
    \bibitem[1]{rikadai}東京理科大学工学部情報工学科 情報工学実験2 2020年度
    東京理科大学工学部情報工学科出版
\end{thebibliography}

\clearpage

\appendix
%%%%%%%%%%%%%%%%%%%%%%%%%%%%%%%%%%%%%%%%%%%%%%%%%%%%%%%%%%%%%%
\end{document}