\documentclass[12pt]{jarticle}
\usepackage{TUSIReport}
\usepackage{otf}
\usepackage{graphicx}
\usepackage{amsmath}
\usepackage{amssymb}
\usepackage{color}
\usepackage{hhline}
\usepackage{fancybox,ascmac}
\usepackage{multirow}
\usepackage{url}
\usepackage{bm}
\usepackage{listings,jlisting}
\lstdefinestyle{log}{
    frame={tblr},
    basicstyle={\footnotesize},
    tabsize={4},
}
\lstdefinestyle{lstR}{
    language={C},
    backgroundcolor={\color[gray]{.85}},
    basicstyle={\small},
    identifierstyle={\small},
    commentstyle={\small\ttfamily \color[rgb]{0,0.5,0}},
    keywordstyle={\small\bfseries \color[rgb]{1,0,0}},
    ndkeywordstyle={\small},
    stringstyle={\small\ttfamily \color[rgb]{0,0,1}},
    frame={tb},
    breaklines=true,
    columns=[l]{fullflexible},
    numbers=left,
    xrightmargin=0zw,
    xleftmargin=3zw,
    numberstyle={\scriptsize},
    stepnumber=1,
    numbersep=1zw,
    morecomment=[l]{//}
}
\begin{document}
%%%%%%%%%%%%%%%%%%%%%%%%%%%%%%%%%%%%%%%%%%%%%%%%%%%%%%%%%%%%%%
% 表紙を出力する場合は,\提出者と\共同実験者をいれる
% \提出者{科目名}{課題名}{提出年}{提出月}{提出日}{学籍番号}{氏名}
% \共同実験者{一人目}{二人目}{..}{..}{..}{..}{..}{八人目}
%%%%%%%%%%%%%%%%%%%%%%%%%%%%%%%%%%%%%%%%%%%%%%%%%%%%%%%%%%%%%%
\提出者{情報工学実験2}{実験テーマ4 統計的推測と単回帰分析}{2020}{9}{28}{4619055}{辰川力駆}
\共同実験者{}{}{}{}{}{}{}{}

%%%%%%%%%%%%%%%%%%%%%%%%%%%%%%%%%%%%%%%%%%%%%%%%%%%%%%%%%%%%%%
% 表紙を出力しない場合は,以下の「\表紙出力」をコメントアウトする
%%%%%%%%%%%%%%%%%%%%%%%%%%%%%%%%%%%%%%%%%%%%%%%%%%%%%%%%%%%%%%
\表紙出力

%%%%%%%%%%%%%%%%%%%%%%%%%%%%%%%%%%%%%%%%%%%%%%%%%%%%%%%%%%%%%%
% 以下はレポート本体である.別途 TeXファイルを作成し \input 使っても良い
%%%%%%%%%%%%%%%%%%%%%%%%%%%%%%%%%%%%%%%%%%%%%%%%%%%%%%%%%%%%%%

\section{はじめに}
本実験では、単回帰分析の考え方と手順を理解することを目標とする。

\section{目的}
\begin{itemize}
    \item [1.]単回帰分析の考え方と手順

          単回帰分析の目的、考え方、手順を理解する
    \item [2.]単回帰分析における行列表現

          単回帰分析における行列表現(線形回帰モデル、正規方程式、最小二乗推定量など)
          を理解する
    \item [3.]実際のデータ解析

          実際のデータに回帰分析を適用することで、解析法を実践的に利用・応用できるようにする
\end{itemize}

\section{実験方法}
\subsection{実験1 単回帰分析の考え方と手順}
6つの市町村の人口と行政職員数の仮想データを表1に示す。
また、各市町村の人口を$x_i$,職員数を$y_i(i=1,...,n(=6))$と表記する。

\begin{table}[htb]
    \begin{center}
        \caption{市町村の人口と行政職員数}
        \begin{tabular}{|c|c|c|} \hline
            市町村 & 人口$x$(千人) & 職員数$y$(人) \\ \hline
            A      & 1             & 10            \\
            B      & 2             & 20            \\
            C      & 3             & 20            \\
            D      & 3             & 40            \\
            E      & 5             & 40            \\
            F      & 1             & 5             \\ \hline
            合計   & 15            & 135           \\\hline
        \end{tabular}
    \end{center}
\end{table}

\begin{itemize}
    \item [1.]次の統計量を計算する。
          \begin{eqnarray}
              \sum_{i=1}^{n} x_i, \sum_{i=1}^{n} y_i, \sum_{i=1}^{n} {x_i}^2, \sum_{i=1}^{n} {y_i}^2, \sum_{i=1}^{n} x_iy_i \nonumber
          \end{eqnarray}
    \item [2.]次式が整理することを証明する。
          \begin{eqnarray}
              \sum_{i=1}^{n} (x_i-\bar{x})^2&=&\sum_{i=1}^{n} {x_i}^2-n\bar{x}^2 \\
              \sum_{i=1}^{n} (y_i-\bar{y})^2&=&\sum_{i=1}^{n} {y_i}^2-n\bar{y}^2 \\
              \sum_{i=1}^{n} (x_i-\bar{x})(y_i-\bar{y})&=&\sum_{i=1}^{n} {x_iy_i}-n\bar{xy}
          \end{eqnarray}
    \item [3.]人口$x$と職員数$y$の基本統計量(データ数、平均、標準偏差、最小値、最大値)を計算する。
          \begin{eqnarray}
              人口xの平均&=&\bar{x}=\frac{\sum_{i=1}^{n} x_i}{n} \nonumber\\
              人口xの標準偏差&=&\sqrt{\frac{\sum_{i=1}^{n} (x_i-\bar{x})^2}{n-1}}=\sqrt{\frac{\sum_{i=1}^{n} {x_i}^2-n\bar{x}^2}{n-1}} \nonumber\\
              職員数yの平均&=&\bar{y}=\frac{\sum_{i=1}^{n} y_i}{n} \nonumber\\
              職員数yの標準偏差&=&\sqrt{\frac{\sum_{i=1}^{n} (y_i-\bar{y})^2}{n-1}}=\sqrt{\frac{\sum_{i=1}^{n} {y_i}^2-n\bar{y}^2}{n-1}} \nonumber
          \end{eqnarray}
    \item [4.]人口$x$と職員数$y$のPearsonの積率相関係数$r$を計算する。
          \begin{eqnarray}
              r=\frac{\sum_{i=1}^{n} (x_i-\bar{x})(y_i-\bar{y})}{\sqrt{\sum_{i=1}^{n} (x_i-\bar{x})^2} \sqrt{\sum_{i=1}^{n} (y_i-\bar{y})^2}}=\frac{\sum_{i=1}^{n} {x_iy_i}-n\bar{xy}}{\sqrt{\sum_{i=1}^{n} (x_i-\bar{x})^2} \sqrt{\sum_{i=1}^{n} (y_i-\bar{y})^2}}
          \end{eqnarray}
    \item [5.]人口$x$を横軸,職員数$y$を縦軸にした散布図を作成して、両者の関係を調べる。
    \item [6.]単回帰モデル$y_i=\beta_0+\beta_1 x_i + \varepsilon_i(i=1,...,n)$をあてはめる。
          $\beta_0$と$\beta_1$の推定量を$\hat{\beta_0}$と$\hat{\beta_1}$とすると、
          目的変数(応答変数)である職員数の予測値は$\hat{y_i}=\hat{\beta_0}+\hat{\beta_1}x_i$で
          与えられる。
          次式の残差平方和$S_e$を$\hat{\beta_0}$と$\hat{\beta_1}$でそれぞれ偏微分する。
          \begin{eqnarray}
              S_e=\sum_{i=1}^{n} (y_i-\hat{y_i})^2 = \sum_{i=1}^{n} (y_i-(\hat{\beta_0}+\hat{\beta_1}x_i))^2
          \end{eqnarray}
    \item [7.]正規方程式を作成する。
    \item [8.]正規方程式を解き、$\beta_0$と$\beta_1$の最小二乗推定量を数式で表現する。
    \item [9.]最小二乗推定量$\hat{\beta_0}$,$\hat{\beta_1}$の値を求める。
    \item [10.]得られた回帰直線($\hat{y_i}=\hat{\beta_0}+\hat{\beta_1}x_i$)を
          手順5で作成した散布図に図示して、結果を考察する。
    \item [11.]データ分析[回帰分析]を用いて、これまでに得られた結果と同様の結果が得られることを確認する。
\end{itemize}

\clearpage

\subsection{実験2 単回帰分析における行列表現}
データ数を$n$とする。
目的変数ベクトル$\bm Y$と説明変数を含む定数行列$\bm X$を次式で定義する。
\[
    \bm Y = \left(
    \begin{array}{c}
            y_1    \\
            y_2    \\
            \vdots \\
            y_n
        \end{array}
    \right)
\]
\[
    \bm X = \left(
    \begin{array}{cc}
            1      & x_1    \\
            1      & x_2    \\
            \vdots & \vdots \\
            1      & x_n
        \end{array}
    \right)
\]
このとき、実験1の単回帰モデルは
\begin{eqnarray}
    \bm y=\bm X \bm \beta+ \bm \varepsilon \nonumber
\end{eqnarray}
で与えられる。ここで$\bm \beta$は母回帰係数,$ \bm \varepsilon$は誤差ベクトルであり、
次式で定義される。

\[
    \bm \beta = \left(
    \begin{array}{c}
            \beta_0 \\
            \beta_1
        \end{array}
    \right)
\]
\[
    \bm \varepsilon = \left(
    \begin{array}{c}
            \varepsilon_0 \\
            \varepsilon_1 \\
            \vdots        \\
            \varepsilon_n
        \end{array}
    \right)
\]
$\bm \beta$の推定量を$\hat{\bm \beta}=(\hat{\beta_0},\hat{\beta_1})^T$とする。
\begin{enumerate}
    \item 残差平方和$S_e$を行列で表現する。
    \item 残差平方和$S_e$を$\hat{\bm \beta}$で微分し、正規方程式を導く。
    \item 正規方程式から最小二乗推定量が
          \begin{eqnarray}
              \hat{\bm \beta}=(\bm X^T \bm X)^{-1} \bm X^T \bm y
          \end{eqnarray}
          で得られることを確認する。
    \item ベクトル$\bm Y$と行列$\bm X$を定義する。
          \clearpage
    \item 次の値を計算する。
          \begin{itemize}
              \item [(a)]$x$の平均$\bar{x}$
              \item [(b)]$y$の平均$\bar{y}$
              \item [(c)]$x$の偏差平方和$\bar{x}=\sum_{i=1}^{n} (x_i-\bar{x})^2$
              \item [(d)]$y$の偏差平方和$\bar{y}=\sum_{i=1}^{n} (y_i-\bar{y})^2$
          \end{itemize}
    \item 次の値を計算する。
          \begin{itemize}
              \item [(a)]最小二乗推定量 $\hat{\bm \beta}=(\bm X^T \bm X)^{-1} \bm X^T \bm Y$
              \item [(b)]予測値$\hat{\bm Y}=\bm X \hat{\bm \beta}$
              \item [(c)]残差$\bm Y-\hat{\bm Y}$
          \end{itemize}
    \item 射影行列$\bm H = \bm X (\bm X^T \bm X)^{-1}\bm X^T$を計算し、次式が成り立つことを確認する。
          \begin{itemize}
              \item [(a)]対称性$\bm H^T=\bm H$
              \item [(b)]べき等性$\bm H \bm H = \bm H$
              \item [(c)]$trace(\bm H)= 2$(パラメータ数)
          \end{itemize}
    \item 得られた最小二乗推定量のもとで、総平方和、モデル平方和、残差平方和を計算し
          \begin{eqnarray}
              総平方和=モデル平方和+残差平方和
          \end{eqnarray}
          が成り立つことを確認する。
    \item 寄与率(決定係数)= モデル平方和/総平方和を計算し、モデルの当てはまりを評価する。
\end{enumerate}

\clearpage

\section{結果・考察・課題}
\subsection{実験1 単回帰分析の考え方と手順}
\begin{shadebox}
    \fbox{課題1}実験1の結果をまとめる。
\end{shadebox}
\begin{itemize}
    \item [1.]計算すると次のようになった。
          \begin{eqnarray}
              \sum_{i=1}^{n} x_i&=&15 \nonumber\\
              \sum_{i=1}^{n} y_i&=&135 \nonumber\\
              \sum_{i=1}^{n} {x_i}^2&=&49 \nonumber\\
              \sum_{i=1}^{n} {y_i}^2&=&4125 \nonumber\\
              \sum_{i=1}^{n} x_iy_i&=&435 \nonumber
          \end{eqnarray}
    \item [2.]式(1),(2),(3)が成り立つことを示す。式(1)の左辺を変形すると、
          \begin{eqnarray}
              \sum_{i=1}^{n} (x_i-\bar{x})^2&=&\sum_{i=1}^{n} (x_i^2-2x_i\bar{x}+\bar{x}^2) \nonumber\\
              &=&\sum_{i=1}^{n} x_i^2-\sum_{i=1}^{n} 2x_i\bar{x}+n\bar{x}^2 \nonumber\\
              &=&\sum_{i=1}^{n} x_i^2-\sum_{i=1}^{n} 2\bar{x}^2+n\bar{x}^2 \nonumber \\
              &=&\sum_{i=1}^{n} x_i^2-2n\bar{x}^2+n\bar{x}^2 \nonumber \\
              &=&\sum_{i=1}^{n} x_i^2-n\bar{x}^2 \nonumber
          \end{eqnarray}
          となり、右辺と一致する。
          同様にして、式(2)の左辺を変形すると、下記のようになり右辺と一致する。
          \begin{eqnarray}
              \sum_{i=1}^{n} (y_i-\bar{y})^2&=&\sum_{i=1}^{n} (y_i^2-2y_i\bar{y}+\bar{y}^2) \nonumber\\
              &=&\sum_{i=1}^{n} y_i^2-2n\bar{y}^2+n\bar{y}^2 \nonumber \\
              &=&\sum_{i=1}^{n} y_i^2-n\bar{y}^2 \nonumber
          \end{eqnarray}
          式(3)も同様の考え方より、
          \begin{eqnarray}
              \sum_{i=1}^{n} (x_i-\bar{x})(y_i-\bar{y})&=& \sum_{i=1}^{n} (x_iy_i-x_i\bar{y}-y_i\bar{x}+\bar{x}\bar{y}) \nonumber \\
              &=& \sum_{i=1}^{n} x_iy_i-n\bar{x}\bar{y}-n\bar{x}\bar{y}+n\bar{x}\bar{y} \nonumber \\
              &=&   \sum_{i=1}^{n} {x_iy_i}-n\bar{xy} \nonumber
          \end{eqnarray}
          となり、証明できた。

    \item [3.]データの数は$x$,$y$どちらも、6つである。残りの基本統計量(平均、標準偏差、最小値、最大値)は以下のようになった。
          \begin{table}[htb]
              \begin{center}
                  \caption{人口と行政職員数の基本統計量}
                  \begin{tabular}{|c|c|c|} \hline
                      基本統計量 & 人口$x$(千人) & 職員数$y$(人) \\ \hline
                      平均       & 2.5           & 22.5          \\
                      標準偏差   & 1.52          & 14.75         \\
                      最小値     & 1             & 5             \\
                      最大値     & 5             & 40            \\ \hline
                  \end{tabular}
              \end{center}
          \end{table}


    \item [4.]Pearsonの積率相関係数$r$は次のようになった。
          \begin{eqnarray}
              r&=&\frac{\sum_{i=1}^{n} (x_i-\bar{x})(y_i-\bar{y})}{\sqrt{\sum_{i=1}^{n} (x_i-\bar{x})^2} \sqrt{\sum_{i=1}^{n} (y_i-\bar{y})^2}} \nonumber\\
              &\fallingdotseq &0.87 \nonumber
          \end{eqnarray}
    \item [5.]人口$x$を横軸,職員数$y$を縦軸にした散布図を作成して、両者の関係を調べる。
    \item [6.]単回帰モデル$y_i=\beta_0+\beta_1 x_i + \varepsilon_i(i=1,...,n)$をあてはめる。
          $\beta_0$と$\beta_1$の推定量を$\hat{\beta_0}$と$\hat{\beta_1}$とすると、
          目的変数(応答変数)である職員数の予測値は$\hat{y_i}=\hat{\beta_0}+\hat{\beta_1}x_i$で
          与えられる。
          次式の残差平方和$S_e$を$\hat{\beta_0}$と$\hat{\beta_1}$でそれぞれ偏微分する。
          \begin{eqnarray}
              S_e=\sum_{i=1}^{n} (y_i-\hat{y_i})^2 = \sum_{i=1}^{n} (y_i-(\hat{\beta_0}+\hat{\beta_1}x_i))^2
          \end{eqnarray}
    \item [7.]正規方程式を作成する。
    \item [8.]正規方程式を解き、$\beta_0$と$\beta_1$の最小二乗推定量を数式で表現する。
    \item [9.]最小二乗推定量$\hat{\beta_0}$,$\hat{\beta_1}$の値を求める。
    \item [10.]得られた回帰直線($\hat{y_i}=\hat{\beta_0}+\hat{\beta_1}x_i$)を
          手順5で作成した散布図に図示して、結果を考察する。
    \item [11.]データ分析[回帰分析]を用いて、これまでに得られた結果と同様の結果が得られることを確認する。
\end{itemize}

\begin{shadebox}
    \fbox{課題2}公表されてるデータ(標本数が50以上)を集めて、回帰分析を適用し、結果を考察する。
\end{shadebox}

\subsection{実験2 単回帰分析における行列表現}
\begin{shadebox}
    \fbox{課題1}実験2の結果をまとめる。
\end{shadebox}

\begin{shadebox}
    \fbox{課題2}行列を用いて統計演算を行う利点を考察する。
\end{shadebox}


\begin{lstlisting}[style=log]
    ぱあ
\end{lstlisting}

\begin{lstlisting}[caption=read\_2\_1byte.c,label=read,style = lstR]
    #include <stdio.h>
\end{lstlisting}


\section{まとめ}
\begin{itemize}
    \item [1.]単回帰分析の考え方と手順を学んだ
          \begin{itemize}
              \item 手計算やエクセルで分析を行った
          \end{itemize}
    \item [2.]単回帰分析における行列表現を学んだ
          \begin{itemize}
              \item 実験1の手順を行列表現した
              \item Rを使い、単回帰分析を行った
          \end{itemize}
\end{itemize}

\section{感想}


\clearpage
% 参考文献
\begin{thebibliography}{99}
    \label{sannkoubunnkenn_chapter}
    \bibitem[1]{rikadai}東京理科大学工学部情報工学科 情報工学実験2 2020年度
    東京理科大学工学部情報工学科出版
\end{thebibliography}

\clearpage

\appendix
%%%%%%%%%%%%%%%%%%%%%%%%%%%%%%%%%%%%%%%%%%%%%%%%%%%%%%%%%%%%%%
\end{document}