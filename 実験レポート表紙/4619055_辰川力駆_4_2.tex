\documentclass[12pt]{jarticle}
\usepackage{TUSIReport}
\usepackage{otf}
\usepackage{graphicx}
\usepackage{amsmath}
\usepackage{amssymb}
\usepackage{color}
\usepackage{hhline}
\usepackage{fancybox,ascmac}
\usepackage{multirow}
\usepackage{url}
\usepackage{listings,jlisting}
\lstdefinestyle{log}{
    frame={tblr},
    basicstyle={\footnotesize},
    tabsize={4},
}
\lstdefinestyle{lstR}{
    language={C},
    backgroundcolor={\color[gray]{.85}},
    basicstyle={\small},
    identifierstyle={\small},
    commentstyle={\small\ttfamily \color[rgb]{0,0.5,0}},
    keywordstyle={\small\bfseries \color[rgb]{1,0,0}},
    ndkeywordstyle={\small},
    stringstyle={\small\ttfamily \color[rgb]{0,0,1}},
    frame={tb},
    breaklines=true,
    columns=[l]{fullflexible},
    numbers=left,
    xrightmargin=0zw,
    xleftmargin=3zw,
    numberstyle={\scriptsize},
    stepnumber=1,
    numbersep=1zw,
    morecomment=[l]{//}
}
\begin{document}
%%%%%%%%%%%%%%%%%%%%%%%%%%%%%%%%%%%%%%%%%%%%%%%%%%%%%%%%%%%%%%
% 表紙を出力する場合は,\提出者と\共同実験者をいれる
% \提出者{科目名}{課題名}{提出年}{提出月}{提出日}{学籍番号}{氏名}
% \共同実験者{一人目}{二人目}{..}{..}{..}{..}{..}{八人目}
%%%%%%%%%%%%%%%%%%%%%%%%%%%%%%%%%%%%%%%%%%%%%%%%%%%%%%%%%%%%%%
\提出者{情報工学実験2}{実験テーマ4 統計的推測と単回帰分析}{2020}{9}{28}{4619055}{辰川力駆}
\共同実験者{}{}{}{}{}{}{}{}

%%%%%%%%%%%%%%%%%%%%%%%%%%%%%%%%%%%%%%%%%%%%%%%%%%%%%%%%%%%%%%
% 表紙を出力しない場合は,以下の「\表紙出力」をコメントアウトする
%%%%%%%%%%%%%%%%%%%%%%%%%%%%%%%%%%%%%%%%%%%%%%%%%%%%%%%%%%%%%%
\表紙出力

%%%%%%%%%%%%%%%%%%%%%%%%%%%%%%%%%%%%%%%%%%%%%%%%%%%%%%%%%%%%%%
% 以下はレポート本体である.別途 TeXファイルを作成し \input 使っても良い
%%%%%%%%%%%%%%%%%%%%%%%%%%%%%%%%%%%%%%%%%%%%%%%%%%%%%%%%%%%%%%

\section{はじめに}
本実験では、単回帰分析の考え方と手順を理解することを目標とする。

\section{目的}
\begin{itemize}
    \item [1.]
\end{itemize}

\begin{lstlisting}
    ぱあ
\end{lstlisting}
\begin{shadebox}
    \fbox{ぱあ}
\end{shadebox}
\begin{lstlisting}[caption=read\_2\_1byte.c,label=read,style = lstR]
    #include <stdio.h>
\end{lstlisting}

\section{実験方法}
\subsection{実験1 推定と検定の考え方と手順}
\begin{itemize}
    \item [1.]
\end{itemize}

\subsection{実験2 $t$分布}
\begin{itemize}
    \item [1.]
\end{itemize}

\section{結果・考察・課題}

\section{まとめ}
\begin{itemize}
    \item [1.]単回帰分析の考え方と手順を学んだ
          \begin{itemize}
              \item 手計算やエクセルで分析を行った
          \end{itemize}
    \item [2.]単回帰分析における行列表現を学んだ
          \begin{itemize}
              \item 実験1の手順を行列表現した
              \item Rを使い、単回帰分析を行った
          \end{itemize}
\end{itemize}

\section{感想}


\clearpage
% 参考文献
\begin{thebibliography}{99}
    \label{sannkoubunnkenn_chapter}
    \bibitem[1]{rikadai}東京理科大学工学部情報工学科 情報工学実験2 2020年度
    東京理科大学工学部情報工学科出版
\end{thebibliography}

\clearpage

\appendix
%%%%%%%%%%%%%%%%%%%%%%%%%%%%%%%%%%%%%%%%%%%%%%%%%%%%%%%%%%%%%%
\end{document}