\documentclass[12pt]{jarticle}
\usepackage{TUSIReport}
\begin{document}
%%%%%%%%%%%%%%%%%%%%%%%%%%%%%%%%%%%%%%%%%%%%%%%%%%%%%%%%%%%%%%
% 表紙を出力する場合は,\提出者と\共同実験者をいれる
% \提出者{科目名}{課題名}{提出年}{提出月}{提出日}{学籍番号}{氏名}
% \共同実験者{一人目}{二人目}{..}{..}{..}{..}{..}{八人目}
%%%%%%%%%%%%%%%%%%%%%%%%%%%%%%%%%%%%%%%%%%%%%%%%%%%%%%%%%%%%%%
\提出者{情報工学実験2}{実験テーマ1 数理計画法}{2020}{9}{7}{4618999}{二階堂ふみ}
\共同実験者{窪田正孝}{唐沢寿明}{菊池桃子}{}{}{}{}{}

%%%%%%%%%%%%%%%%%%%%%%%%%%%%%%%%%%%%%%%%%%%%%%%%%%%%%%%%%%%%%%
% 表紙を出力しない場合は,以下の「\表紙出力」をコメントアウトする
%%%%%%%%%%%%%%%%%%%%%%%%%%%%%%%%%%%%%%%%%%%%%%%%%%%%%%%%%%%%%%
\表紙出力

%%%%%%%%%%%%%%%%%%%%%%%%%%%%%%%%%%%%%%%%%%%%%%%%%%%%%%%%%%%%%%
% 以下はレポート本体である.別途 TeXファイルを作成し \input 使っても良い
%%%%%%%%%%%%%%%%%%%%%%%%%%%%%%%%%%%%%%%%%%%%%%%%%%%%%%%%%%%%%%

\section{実験の要旨}
何について,どのような結果が得られたかを,できるだけ簡潔に書く.
これを読むだけで,この報告書に何が書かれているか分かるように書くべきである.

\section{実験の目的}
この実験は何のために行ったか,簡潔にその目的を書く.

\section{実験の原理(理論)}
テキスト,その他書物の丸写しをせずに,簡潔にまとめる.

\section{実験装置あるいは実験方法}
実験に使用した器具やサンプルなどは,それぞれ,表にすると分かりやすい.
実験方法は,要領よく箇条書きにするなどして,簡潔に書く.

\section{結果}
実験結果は,その性質をよく考えて,表 または 図 (グラフ) にする.
グラフの場合は,縦軸や横軸が何を示すかを明記する.
表の数値などは,有効数字に留意する.

\section{検討・考察}
検討・考察では,実験によって得られた客観的事実に基づいて,
報告者が論理的な思考を展開し,得られた見解を,分かりやすく簡潔に述べる.

\section{結論}
主要な結果と,検討の結果明確になったことを,簡潔に表現する.

% 参考文献
\begin{thebibliography}{99}
\label{sannkoubunnkenn_chapter}
 \bibitem{collins}
	 J. J. Collins et al.,
	 {\em PRE}, {\bf 52}(4):R3321, 1995.
 \bibitem{izh} E. M. Izhikevich, 
	 {\em IEEE Trans. NN}, {\bf 14}(6):1569, 2003. 
\end{thebibliography}

\clearpage
% 付録
\appendix
\section{付録}
%%%%%%%%%%%%%%%%%%%%%%%%%%%%%%%%%%%%%%%%%%%%%%%%%%%%%%%%%%%%%%
\end{document}