\documentclass[12pt]{jarticle}
\usepackage{TUSIReport}
\usepackage{bm}
\usepackage{ascmac}
\usepackage{framed}
\usepackage[dvipdfmx]{graphicx}
\begin{document}
\提出者{情報工学実験2}{実験テーマ1 数理計画法}{2020}{9}{21}{4619060}{照永詩恩}
\共同実験者{}{}{}{}{}{}{}{}
\表紙出力


\section{結果}
\begin{description}
    \item[(2)]
    \item[問5]
          
    \item[問6]
    \item[(3)]
          「とある学校で6科目中3科目のテストを受けなければいけない.事前にそれぞれの科目はどのくらい得意なのかというアンケートを取りさらに
          問題作成者に難易度はどれくらいかを聞いたという設定にする.Aくんは難易度が150を超えてしまうと精神状態が悪くなり今までの力が発揮できなくなってしまうようです.
          それ以下であれば得意であれば得意であるほど点数が高くなるようです.どの科目を選ぶのが最適でアンケートで答えた数値の合計はいくつになるか」という問題を作成した.
          \begin{table}[h]
              \caption{Aくんがそれぞれどれくらい得意なのかとその科目のテストの難易度}
              \begin{center}
                  \begin{tabular}{|c|c|c|c|c|c|c|}
                      \hline
                                   & $x_1$ & $x_2$ & $x_3$ & $x_4$ & $x_5$ & $x_6$ \\
                      \hline
                      得意の度合い & 10    & 8     & 9     & 4     & 5     & 1     \\
                      \hline
                      難易度       & 80    & 60    & 70    & 50    & 40    & 10    \\
                      \hline
                  \end{tabular}
              \end{center}
          \end{table}
          これを定式化してみる.\\
          決定変数は受けるか受けないかということでバイナリ変数$\boldsymbol{x}$とする.目的関数と制約条件をそれぞれ,
          \begin{eqnarray}
              f(\boldsymbol{x})&=&10x_1+8x_2+9x_3+4x_4+5x_5+x_6\nonumber\\
              S&=&\{\boldsymbol{x};x_1,x_2,x_3,x_4,x_5,x_6\in \{0,1\},\nonumber\\
              &&x_1+x_2+x_3+x_4+x_5+x_6=3,\nonumber\\
              &&80x_1+60x_2+70x_3+50x_4+40x_5+10x_6\leq 150\}\nonumber
          \end{eqnarray}
          と書くことができ,Maximize \ $f(x)$ \ subject\ to\ $\boldsymbol{y}\in$ S とかく.       
\end{description}
\section{検討・考察}
\begin{description}
    \item[(2)]
    \item[問1]
    \item[問2]
          
    \item[問3]
          
    \item[問4]
    \item[問5]
    \item[問6]
    \item[(3)]
          バイナリ変数を決定変数とする問題を作成してみた.私自身も一般教養を選択するにおいて前年度の成績の比率などを参考にしている.そのような経験から
          このような問題を作る発想に至った.  
\end{description}
\section{結論}
\begin{thebibliography}{99}
    \label{sannkoubunnkenn_chapter}
    \bibitem{rikadai}
    情報工学実験1,
    東京理科大学 工学部 情報工学科, 2020. 
\end{thebibliography}
\clearpage
% 付録
\appendix
\section{付録}
\begin{framed}
    \begin{verbatim}
#include<stdio.h>
int main()
{
	int a[5]={10,5,7,6,3};//嬉しさ
	int b[5]={140,80,130,70,30};//値段
	int c=1;//論理積につかう値
	int d[32][5];//各セルに対して0か1を格納するための配列
	int i,j;//for文に使う
	int k;//論理積の結果を代入する用
	int ans1,ans2;//嬉しさと値段の合計
	int e=-100000000,f=300;//嬉しさの最大値と制約条件である300円
	int g;//10進数表記用
			
	for(i=0;i<32;i++)//配列dに1~32までの二進数の各ビットの数字を格納
	{
		k=i+1;
		for(j=4;j>=0;j--)
		{
			d[i][j]=k&c;
			k=k >> 1;
		}
	}
			
	for(i=0;i<32;i++)//嬉しさと値段の合計を求めて最適値を出す
	{
		ans1=0;
		ans2=0;
		for(j=0;j<5;j++)
		{
 			ans1+=a[j]*d[i][j];
			ans2+=b[j]*d[i][j];
		}
		if(ans1>e&&ans2<=f&&ans1>0)
		{
			e=ans1;
			g=i+1;
		}
	}
			
	printf("最適値%d(%dのとき)\n",e,g);
	return 0;
}
    \end{verbatim}
\end{framed}
\begin{center}
    図A.2:(2)の問3ソースコード
\end{center}
\end{document}